%% Define DUNE FD DAQ terms.
%% This needs merging eventually into the TDR glossary.
%% For now it directly uses low-level glossaries macros.

\newglossary*{daq}{DAQ Glossary}

\makenoidxglossaries

% These are similar to dune-words but put in the "daq:" namespace.
% You must remember to add this prefix when referencing DAQ terms.
% Eg, \gls{daq:node}.


%          1     2    3     4
% \daqterm[]{label}{term}{desc}
\newcommand\daqterm[4][]{%
  \newglossaryentry{daq:#2} {%
    type=daq,
    text={#3},
    long={#3},
    name={\glsentrylong{daq:#2}},
    description={#4},
    #1
  }}

%          1     2      3     4   5
% \daqabbr[]{label}{abbrev}{term}{desc}
\newcommand\daqabbr[5][]{%
  \newglossaryentry{daq:#2} {%
    type=daq,
    text={#3},
    long={#4},
    shortplural={{#3}\glspluralsuffix},
    longplural={{#4}\glspluralsuffix{}},
    name={\glsentrylong{daq:#2}{} (\glsentrytext{daq:#2}{})},
    first={\glsentryname{daq:#2}},
    firstplural={\glsentrylong{daq:#2}\glspluralsuffix{} (\glsentrytext{daq:#2}\glspluralsuffix{})},
    description={#5},
    #1
  }}

%% From top down

\daqabbr{daq}{DAQ}{Data Acquisition}{A distributed, network application and the hardware on which it operates. 
  It is composed of various components which ingest data from detector electronics. 
  Based on the content of that data as well as human and automated guidance and configuration information it produces output in the form of files which contain a small subset of the input data.}

\daqterm{part}{partition} {A synonym for \gls{daq:instance}.   Different DAQ partitions do not communicate nor inter-operate in any direct manner.  A partition may be considered to be composed of multiple \glspl{daq:frag}.}

\daqterm{instance}{instance}{A synonym for \gls{daq:part}. A wholly independent instance of a DAQ.}

%% split a DAQ "horizontally"

\daqabbr{fe}{FE}{front-end}{A general description for the portion of the DAQ subsystems which are upstream near the source of input data provided by detector electronics.}

\daqabbr{be}{BE}{back-end}{A general description for the portion of the DAQ subsystems which are downstream near the sink of output data provided by the offline.}

%% split a DAQ "vertically"

\daqterm[see={daq:datafrag,daq:daqfrag,daq:detfrag}]{frag}{fragment}{A general description for the largest subdivision of an aggregate for which data corresponding to that subdivision is treated in an atomic manner for the purposes of \gls{daq:ds}.  This term is often used qualified in some manner to refer to fragment of a specific larger aggregate.  The fragment is understood to correspond to the same lines of subdivision in all such qualified uses.}

\daqterm{daqfrag}{fragment}{The portion of DAQ components corresponding to one \gls{daq:frag}.}

\daqterm{datafrag}{data fragment}{A subset of the \gls{daq:datastream} specified as a spanning a given \gls{daq:window} and originating from one \gls{daq:detfrag}.}

\daqterm{detfrag}{detector fragment}{A subset of detector electronics which provides data corresponding to one \gls{daq:frag}.}

%% Major functionality

\daqabbr[see={daq:selftrig,daq:td}]{ds}{DS}{Data Selection}{The action of providing input into the \gls{daq:td} typically by examining the \gls{daq:datastream} for activity.  As a subsystem, it refers to the \glspl{daq:component} for performing the action.}

\daqabbr[see={daq:ptl}]{td}{TD}{trigger decision}{The action of aggregating all input from  \gls{daq:ds} and mapping it to a \gls{daq:ta}.}

\daqabbr{da}{DA}{data aggregation}{The execution of a \gls{daq:td} performed by retrieving indicated \glspl{daq:datafrag} from their corresponding \gls{daq:buf}, collecting them into one \gls{daq:datablock} available for \gls{daq:beo}.}

\daqabbr{rc}{RC}{Run Control}{The action by which human or automated intention is applied to DAQ \glspl{daq:component} for the purpose of initiating, configuring, directing, modifying or terminating their actions.}

\daqterm{readout}{readout}{A generic term used to describe the transfer of data out of one system and into another.}

\daqabbr{fer}{FER}{Front-end Readout}{The portion of the DAQ that receives data from detector electronics.}

\daqabbr{beo}{BEO}{back-end output}{The portion of the DAQ that writes output files to disk buffer shared with offline.}

\daqterm{buf}{primary buffer}{A distributed buffer through which all detector data shall stream such that the data may be consumed following some minimum delay long enough to provide time for the \gls{daq:ds}, \gls{daq:td} and \gls{daq:da} processes to complete. 
  Each atomic unit of buffer shall be associated with an integral number of \glspl{daq:frag}.}

%% Roles

\daqabbr[see={daq:mtl,daq:gtl}]{ptl}{PTL}{pinnacle trigger logic}{Generic term for the DAQ \gls{daq:role} which performs \gls{daq:td} to produce a stream of \glspl{daq:ta}.}

\daqabbr{mtl}{MTL}{Module Trigger Logic}{A \gls{daq:component} acting as a \gls{daq:ptl} at the level of one detector module.  In a DAQ with a MTL, all modules operate as peers and there is no single, central \gls{daq:ptl}.  However, they may indirectly send module level \gls{daq:td} to a central \gls{daq:etl} which may forward the information to other \glspl{daq:mtl} in the form of a \gls{daq:tc}.}

\daqabbr{etl}{ETL}{External Trigger Logic}{A \gls{daq:component} which acts as an intermediate between \glspl{daq:mtl}.}

\daqabbr{gtl}{GTL}{Global Trigger Logic}{A \gls{daq:component} acting as a \gls{daq:ptl} for the entire detector.  In a DAQ with a GTL, no \glspl{daq:mtl} nor \gls{daq:etl} exist.}

\daqabbr{eb}{EB}{Event Builder}{A component in the \gls{daq:be} of the DAQ which performs \gls{daq:da}.}

\daqterm{dsc}{Data Selector}{A component in the \gls{daq:fe} of the DAQ which services queries for \glspl{daq:datafrag} from its associated \gls{daq:buf}.}

%% Generic part names

\daqterm{component}{component}{A logical or physical subdivision of the DAQ which performs a specific \gls{daq:role}.  It may refer to a software processes or a unit of hardware.}

\daqterm{node}{node}{An asynchronous software agent implementing a \gls{daq:component} which participates in the \gls{daq:ipc} system through the exchange of \glspl{daq:msg} with other nodes.}

\daqterm{role}{role}{A action, behavior or resource which makes up some portion of the DAQ.  Although abstract, a role has specified, required characteristics which must be provided for any satisfactory implementation of the role.}

%% Comms

\daqabbr{ipc}{IPC}{interprocess communication}{A mechanism by which DAQ \glspl{daq:component} exchange \glspl{daq:msg}.}

\daqterm{msg}{message}{A discrete, well formed unit of information produced or consumed by a DAQ \gls{daq:node} as part of the \gls{daq:ipc} system.}

\daqabbr{ta}{TA}{Trigger Assertion}{A \gls{daq:msg} that describes a subset of the \gls{daq:datastream} to be output by the DAQ. 
  It consists of a description for a \gls{daq:datablock} as collection of descriptions of \glspl{daq:datafrag} and for each a number of \glspl{daq:actdesc}.  (Note: this is often just called ``trigger'' but that shall be resisted.)}

\daqabbr{tc}{TC}{Trigger Candidate}{A \gls{daq:msg} which describes a portion of the detector data stream to be considered for output or to lead to output of other portions by the \gls{daq:td}.}

\daqabbr{tp}{TP}{Trigger Primitive}{A \gls{daq:msg} which describes potential activity over some  contiguous period of time and from a single electronics channel.  These may be combined across some number of contiguous channels to produce \glspl{daq:tc}.}


\daqterm{tps}{trigger primitive source}{A DAQ \gls{daq:role} performed by a DAQ \gls{daq:node} which produces a stream of \gls{daq:tp} \glspl{daq:msg}.}


\daqterm{tcs}{trigger candidate source}{A DAQ \gls{daq:role} performed by a DAQ \gls{daq:node} which produces a stream of \gls{daq:tc} \glspl{daq:msg}.}


\daqterm{selftrig}{self-trigger}{A process by which the DAQ may select a portion of its input data stream by examining a portion of its input data stream. 
  The selected portion need not be a strict subset of the examined portion.}



\daqterm{datablock}{Data Block}{A subset of the \gls{daq:datastream} specified in terms of a \gls{daq:window} and a set of \glspl{daq:datafrag}.  This term may also refer to the period and addressing associated with its boundary in time and channel.  (Note: it is tempting to call this an ``event'' but that shall be resisted.)}


\daqterm[see={daq:ta}]{actdesc}{Activity Descriptor}{One of an enumerated list of labels used to indicate a possible type of activity is present in some examined data.}


\daqterm{datastream}{detector data stream}{The continuous data input to one DAQ \gls{daq:part} from one \gls{daq:detpart}.}

\daqterm{detpart}{detector partition}{The portion detector electronics and other data sources associated with one \gls{daq:part}.}


\daqterm{window}{time window}{A subset of the \gls{daq:datastream} which spans a single, contiguous period of time.}
